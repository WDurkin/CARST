\documentclass[12pt]{article}
\usepackage{graphicx}
\usepackage{longtable}
\usepackage{lineno}
\usepackage{natbib}
\usepackage[margin=1in]{geometry}

\newenvironment{myindentpar}[1]%
{\begin{list}{}%
 {\setlength{\leftmargin}{#1}}%
  \item[]%
 }
{\end{list}}

\title{Obtaining $\frac{dh}{dt}$ Using ``weightedRegression.py''}

\author{Andrew Kenneth Melkonian}
\begin{document}

\maketitle

\vspace{2pt}
\hrule
\vspace{6pt}
\noindent {\bf weightedRegression.py}
\vspace{6pt}
\hrule
\vspace{16pt}

\noindent {\bf \underline{Description}:} \\

``weightedRegression.py'' reads a 5-column ASCII text file of longitudes, latitudes, elevations, dates (in decimal years) and  uncertainties (same units as elevations) for individual pixels.
Pixels are separated by a line containing only the greater than character (``$>$'').
The script applies a weighted linear regression on a pixel-by-pixel basis, excluding elevations that deviate by more than the upper and `lower thresholds provided by the user. \\

\noindent {\bf \underline{Dependencies}:} \\

\noindent {\bf Python} (with numpy and scipy) \\

\noindent {\bf \underline{Usage}:} \\

\noindent python weightedRegression.py /path/to/elevs.txt output\_label min\_year max\_year upper\_threshold lower\_threshold reference\_type \\

\noindent {\bf \underline{Example}:} \\

\noindent python weightedRegression.py /path/to/elevs.txt glacier 1999 2015 5 10 first \\

\noindent {\bf \underline{Input Parameters}:} \\

\noindent {\bf /path/to/elevs.txt:} Path to five-column ASCII text file, first column is longitude/UTM easting, second is latitude/UTM northing, third is elevation, fourth is date in decimal years, and fifth is elevation uncertainty (typically standard deviation of off-ice elevation differences with some reference surface).
Each pixel/geographic point should be separated by a line with only the "$>$" character (like a GMT polygon file). \\

\noindent {\bf output\_label:} String used to determine name of output file (created in current directory).
If ``output\_label'' is ``glacier'', ``upper\_threshold'' is ``5'', and ``lower\_threshold'' is ``10'', then the file created will be ``glacier\_p5m10.txt''. \\

\noindent {\bf min\_year:} Number, only dates more recent than ``min\_year'' will be considered. \\

\noindent {\bf max\_year:} Number, only dates earlier than ``max\_year'' will be considered. \\

\noindent {\bf upper\_threshold:} Maximum allowed positive deviation (moving forward in time) from reference elevation in meters per year.
For example, if the reference elevation at a point is 150 m in 2005.00 and ``upper\_threshold'' is ``5'', then an elevation from 2006.00 must be less than 155 m and an elevation from 2004.00 must be greater than 145 m.
This must be a valid number. \\

\noindent {\bf lower\_threshold:} Maximum allowed negative deviation (moving forward in time) from reference elevation in meters per year.
For example, if the reference elevation at a point is 150 m in 2005.00 and ``lower\_threshold'' is ``10'', then an elevation from 2006.00 must be greater than 140 m and an elevation from 2004.00 must be less than 160 m.
This must be a valid number. \\

\noindent {\bf reference\_type:} Must be ``first'', ``median'' or ``lowest'', to indicate that reference elevation should be the first (in time), median elevation, or elevation with the lowest uncertainty, respectively.
This parameter does not need to be provided, the default is ``first''. \\

\noindent {\bf \underline{Output}:} \\

\noindent The file created by the script (in the directory it is run) is named according to the input parameters.
For example, if ``output\_label'' is ``glacier'', ``upper\_threshold'' is ``5'', and ``lower\_threshold'' is ``10'', then the file created will be ``glacier\_p5m10.txt''.
Each row corresponds to the results for a single pixel.
This file has 20 columns: \\
1) Longitude/UTM easting \\
2) Latitude/UTM northing \\
3) Surface elevation change rate ($\frac{dh}{dt}$), typically m yr\textsuperscript{-1} \\
4) Surface elevation change rate uncertainty (one standard deviation) \\
5) Intercept \\
6) Intercept uncertainty \\
7) Number of elevations incorporated into the regression \\
8) Date of median elevation incorporated into the regression (decimal years) \\
9) Elevation of median elevation incorporated into the regression \\
10) Uncertainty of median elevation incorporated into the regression \\
11) Date of first elevation incorporated into the regression (in time, decimal years) \\
12) Elevation of first elevation incorporated into the regression (in time) \\
13) Uncertainty of first elevation (in time) incorporated into the regression \\
14) Date of last elevation incorporated into the regression (in time, decimal years) \\
15) Last elevation (in time) incorporated into the regression \\
16) Uncertainty of last elevation (in time) incorporated into the regression \\
17) Date of reference elevation (decimal years) \\
18) Elevation of reference elevation \\
19) Uncertainty of reference elevation \\
20) Interval between first and last elevation incorporated into the regression (in decimal years) \\


\end{document}
